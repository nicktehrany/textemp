\documentclass{article}
\usepackage{graphicx}
\usepackage[hyphens]{url}
\usepackage{amsmath}
\usepackage{xcolor}
\usepackage{listings}

\definecolor{codegreen}{rgb}{0,0.6,0.1}
\definecolor{codegray}{rgb}{0.2,0.2,0.2}
\definecolor{backcolour}{rgb}{0.95,0.95,0.92}

\lstdefinestyle{mystyle}{
    backgroundcolor=\color{backcolour},   
    commentstyle=\color{codegreen},
    keywordstyle=\color{magenta},
    numberstyle=\tiny\color{codegray},
    basicstyle=\ttfamily\footnotesize,
    breakatwhitespace=false,         
    breaklines=true,                
    keepspaces=true,                 
    numbers=left,                    
    numbersep=5pt,                  
    showspaces=false,                
    showstringspaces=false,
    showtabs=false,                  
    tabsize=2
}

\lstset{style=mystyle}

      \textwidth 15cm
      \textheight 22cm
      \parindent 10pt
      \oddsidemargin 0.85cm
      \evensidemargin 0.37cm

\title{Assignment 1 \\IN4191: Security and Cryptography}
\author{Nick-Andian Tehrany (5411726)}
\date{\today}

\begin{document}

\maketitle

% Use \textbf{Exercise 1.} for less obnoxious exercise headings
\section{Exercise 1.}
\begin{enumerate}
    \item [a)] Here we have the answers for exercise 1a, with some code in~\ref{appendix:python_code}.

    \item [b)] As Einstein said,
          \begin{equation*}
              \begin{split}
                  E = mc^2
              \end{split}
          \end{equation*}
\end{enumerate}

\section{Exercise 2.}
\begin{enumerate}
    \item [a)]  Exercise 2 will be here. An interesting read on non-volatile memory~\cite{inproceedings:pmemguide}.
\end{enumerate}

\bibliography{main}
\bibliographystyle{abbrv}

\appendix
\section{Python Code}
\label{appendix:python_code}
\begin{lstlisting}[language=Python, caption={\raggedright Some python Code.}]
#!/bin/bash

print("This is some python code.")
\end{lstlisting}

\end{document}